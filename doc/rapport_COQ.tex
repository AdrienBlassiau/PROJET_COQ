\documentclass{report}
\usepackage{natbib}
\usepackage[utf8]{inputenc}
\usepackage[french]{babel}
\usepackage[T1]{fontenc}
\usepackage{eurosym}
\usepackage{fullpage}
\usepackage{coqdoc}
\usepackage{chngcntr}
\usepackage{amsmath,amssymb}
\usepackage{url}
\usepackage[hmargin=2cm, vmargin=2cm]{geometry}
\usepackage{graphicx}
\usepackage{color}
\usepackage[table,xcdraw]{xcolor}
\usepackage{listings}
\lstset{breaklines=true}
\usepackage{float}
\usepackage{hyperref}
\usepackage[table,xcdraw]{xcolor}

\usepackage{url}
\usepackage{lmodern}
\usepackage{minted}
\usepackage{array}
\usepackage{fancyvrb}
\usepackage{alltt}
\usepackage{amsmath}
\usepackage{caption}
\usepackage{lscape}
\usepackage{tikz}
\usepackage{varwidth}
\usepackage{listings}
\usepackage{calc}
\usetikzlibrary{shapes,arrows,positioning,calc}

\newsavebox{\BBbox}
\newenvironment{DDbox}[1]{
\begin{lrbox}{\BBbox}\begin{minipage}{\linewidth}}
{\end{minipage}\end{lrbox}\noindent\colorbox{white}{\usebox{\BBbox}} \\
[.5cm]}


\counterwithin*{section}{part}
\renewcommand*\rmdefault{pag}

\newcommand{\HorizontalLine}{\rule{\linewidth}{0.3mm}}

\makeatletter
\renewcommand{\thesection}{\@arabic\c@section}
\makeatother

\definecolor{bg}{rgb}{0.95,0.95,0.95}
\newminted{ocaml}{bgcolor=bg,fontsize=\large,linenos=true}

\definecolor{gr}{HTML}{C0C0C0}
\definecolor{no}{HTML}{000000}
\definecolor{or}{HTML}{F66B03}
\definecolor{vf}{HTML}{016401}
\definecolor{ro}{HTML}{FF0002}
\definecolor{sa}{HTML}{FFCCC9}
\definecolor{bf}{HTML}{00019B}
\definecolor{ja}{HTML}{FCFF31}
\definecolor{vi}{HTML}{6201C9}
\definecolor{bc}{HTML}{31FFF8}

%fontfamily=tt,
%numberblanklines=true,
%numbersep=5pt,
%gobble=0,
%frame=leftline,
%framerule=0.4pt,
%framesep=2mm,
%funcnamehighlighting=true,
%tabsize=4,
%obeytabs=false,
%mathescape=false
%samepage=false,
%showspaces=false,
%showtabs =false,
%texcl=false,

\definecolor{bash}{rgb}{0.95,0.95,0.95}
\newminted{bash}{bgcolor=bash,fontsize=\large,linenos=true}

%fontfamily=tt,
%numberblanklines=true,
%numbersep=5pt,
%gobble=0,
%frame=leftline,
%framerule=0.4pt,
%framesep=2mm,
%funcnamehighlighting=true,
%tabsize=4,
%obeytabs=false,
%mathescape=false
%samepage=false,
%showspaces=false,
%showtabs =false,
%texcl=false,

\definecolor{myred}{RGB}{229, 34, 47}

\title{Rapport du DM de Coq}
\author{Adrien BLASSIAU\\Corentin JUVIGNY}
\date{\today}

\begin{document}
\maketitle

Avant de commencer les deux exercices, on importe tout d'abord le module \texttt{List} comme demandé. On importe aussi \texttt{Omega} qui fournit la tactique \textbf{omega} qui se révèle être très pratique pour la gestion des expressions arithmétiques. À noter que l'on peut très bien faire sans (en utilisant des lemmes du module \texttt{Arith.Gt}, \texttt{Arith.Lt}, etc) mais cela permet de réduire de quelques lignes les preuves ...\\\\
\noindent Les théorèmes admis (donc ceux que nous n'avons pas réussi à démontrer) sont notés en \textcolor{red}{rouge}.

\section{Exercice 1 : Listes et comptage}

Dans la suite on travaille avec un type A quelconque muni d’une égalité décidable. Pour les tests, on utilise nat (ils sont commentés dans le fichier .v).
\begin{coqdoccode}
\coqdocemptyline
\coqdocnoindent
\coqdockw{Variable} \coqdocvar{A} : \coqdockw{Type}.\coqdoceol
\coqdocemptyline
\coqdocnoindent
\coqdockw{Hypothesis} \coqdocvar{dec\_A} : \coqdockw{\ensuremath{\forall}} (\coqdocvar{x} \coqdocvar{y} : \coqdocvar{A}), (\{\coqdocvar{x}=\coqdocvar{y}\}+\{\~{}\coqdocvar{x}=\coqdocvar{y}\}).\coqdoceol
\coqdocemptyline
\end{coqdoccode}

\paragraph{Question 1}

occ prend en argument un élément x de type A et une liste l de type (list A) et retourne le nombre d’occurrences de x dans l. Elle a la signature suivante :

\noindent
\noindent\begin{coqdoccode}
\coqdocemptyline
\coqdocnoindent
\coqdocvar{occ} :
\coqdocvar{A} \ensuremath{\rightarrow}
\coqdocvar{list} \coqdocvar{A} \ensuremath{\rightarrow} \coqdocvar{nat}
\coqdocemptyline
\end{coqdoccode}

\paragraph{Question 2}

On démontre le théorème suivant :

\noindent\begin{coqdoccode}
\coqdocemptyline
\coqdocnoindent
\coqdockw{Theorem} \coqdocvar{occ\_app} : \coqdockw{\ensuremath{\forall}} (\coqdocvar{x} : \coqdocvar{A}) \coqdocvar{l1} \coqdocvar{l2},
\coqdocvar{occ} \coqdocvar{x} (\coqdocvar{app} \coqdocvar{l1} \coqdocvar{l2}) = (\coqdocvar{occ} \coqdocvar{x} \coqdocvar{l1}) + (\coqdocvar{occ} \coqdocvar{x} \coqdocvar{l2}).\coqdoceol
\coqdocemptyline
\end{coqdoccode}

\paragraph{Question 3}

On démontre le théorème suivant :

\noindent\begin{coqdoccode}
\coqdocemptyline
\coqdocnoindent
\coqdockw{Theorem} \coqdocvar{occ\_filter} : \coqdockw{\ensuremath{\forall}} (\coqdocvar{P} : \coqdocvar{A} \ensuremath{\rightarrow} \coqdocvar{bool}) (\coqdocvar{a} : \coqdocvar{A}) \coqdocvar{l},
\coqdocnoindent
\coqdocvar{occ} \coqdocvar{a} (\coqdocvar{filter} \coqdocvar{P} \coqdocvar{l}) = \coqdockw{if} (\coqdocvar{P} \coqdocvar{a}) \coqdockw{then} \coqdocvar{occ} \coqdocvar{a} \coqdocvar{l} \coqdockw{else} 0.\coqdoceol
\coqdocemptyline
\end{coqdoccode}

\paragraph{Question 4}

On démontre le théorème suivant :

\noindent\begin{coqdoccode}
\coqdocemptyline
\coqdocnoindent
\coqdockw{Theorem} \coqdocvar{occ\_map} : \coqdockw{\ensuremath{\forall}} (\coqdocvar{y}:\coqdocvar{A}) (\coqdocvar{z}:\coqdocvar{A}), \coqdocvar{y}\ensuremath{\not=}\coqdocvar{z} \ensuremath{\rightarrow} \coqdoctac{\ensuremath{\exists}} (\coqdocvar{f} : \coqdocvar{A} \ensuremath{\rightarrow} \coqdocvar{A}) (\coqdocvar{x} : \coqdocvar{A}) (\coqdocvar{l} : \coqdocvar{list} \coqdocvar{A}),
\coqdocnoindent
\coqdocvar{occ} (\coqdocvar{f} \coqdocvar{x}) (\coqdocvar{map} \coqdocvar{f} \coqdocvar{l}) \ensuremath{\not=} \coqdocvar{occ} \coqdocvar{x} \coqdocvar{l}.\coqdoceol
\coqdocemptyline
\end{coqdoccode}

\paragraph{Question 5}

On démontre le théorème de l'énoncé en deux temps, en utilisant un petit lemme intermédiaire :

\noindent\begin{coqdoccode}
\coqdocemptyline
\coqdocnoindent
\coqdockw{Lemma} \coqdocvar{mem\_diff} : \coqdockw{\ensuremath{\forall}} (\coqdocvar{l} : \coqdocvar{list} \coqdocvar{A}) (\coqdocvar{x} : \coqdocvar{A}) (\coqdocvar{y} : \coqdocvar{A}), \coqdocvar{mem} \coqdocvar{x} (\coqdocvar{cons} \coqdocvar{y} \coqdocvar{l}) \ensuremath{\rightarrow} \coqdocvar{x} \ensuremath{\not=} \coqdocvar{y} \ensuremath{\rightarrow}
\coqdocnoindent
\coqdocvar{mem} \coqdocvar{x} \coqdocvar{l}.\coqdoceol
\coqdocemptyline

\coqdocnoindent
\coqdockw{Theorem} \coqdocvar{mem\_null\_1} : \coqdockw{\ensuremath{\forall}} (\coqdocvar{l} : \coqdocvar{list} \coqdocvar{A}) (\coqdocvar{x} : \coqdocvar{A}), \coqdocvar{occ} \coqdocvar{x} \coqdocvar{l} = 0 \ensuremath{\rightarrow} \~{}(\coqdocvar{mem} \coqdocvar{x} \coqdocvar{l}).\coqdoceol
\coqdocemptyline

\coqdocnoindent
\coqdockw{Theorem} \coqdocvar{mem\_null\_2} : \coqdockw{\ensuremath{\forall}} (\coqdocvar{l} : \coqdocvar{list} \coqdocvar{A}) (\coqdocvar{x} : \coqdocvar{A}), \~{}(\coqdocvar{mem} \coqdocvar{x} \coqdocvar{l}) \ensuremath{\rightarrow} \coqdocvar{occ} \coqdocvar{x} \coqdocvar{l} = 0.\coqdoceol
\coqdocemptyline
\end{coqdoccode}


\paragraph{Question 6}

On démontre le théorème de l'énoncé en deux temps. Nous ne sommes pas parvenus à démontrer l'un des sens du théorème :

\noindent\begin{coqdoccode}
\coqdocemptyline
\coqdocnoindent
\coqdockw{Theorem} \coqdocvar{doublon\_1} : \coqdockw{\ensuremath{\forall}} (\coqdocvar{l} : \coqdocvar{list} \coqdocvar{A}) (\coqdocvar{x} : \coqdocvar{A}), \coqdocvar{nodup} \coqdocvar{l} \ensuremath{\rightarrow} \coqdocvar{occ} \coqdocvar{x} \coqdocvar{l} \ensuremath{\le} 1.\coqdoceol
\coqdocemptyline
\end{coqdoccode}

\begin{coqdoccode}

\coqdocnoindent
\textcolor{red}{
\coqdockw{Theorem} \coqdocvar{doublon\_2} : \coqdockw{\ensuremath{\forall}} (\coqdocvar{l} : \coqdocvar{list} \coqdocvar{A}) (\coqdocvar{x} : \coqdocvar{A}), \coqdocvar{occ} \coqdocvar{x} \coqdocvar{l} \ensuremath{\le} 1 \ensuremath{\rightarrow} \coqdocvar{nodup} \coqdocvar{l}.\coqdoceol
\coqdocemptyline
\coqdocemptyline}
\end{coqdoccode}

\section{Exercice 2 : Implantation des multi-ensembles}

Dans la suite on travaille avec un type T quelconque muni d’une égalité décidable. Pour les tests, on utilise nat (ils sont commentés dans le fichier .v).

\begin{coqdoccode}
\coqdocemptyline
\coqdocnoindent
\coqdockw{Variable} \coqdocvar{T} : \coqdockw{Type}.\coqdoceol
\coqdocemptyline
\coqdocnoindent
\coqdockw{Hypothesis} \coqdocvar{T\_eq\_dec} : \coqdockw{\ensuremath{\forall}} (\coqdocvar{x} \coqdocvar{y} : \coqdocvar{T}), \{\coqdocvar{x}=\coqdocvar{y}\} + \{\~{}\coqdocvar{x}=\coqdocvar{y}\}.\coqdoceol
\coqdocemptyline

\end{coqdoccode}

\subsection{Implantation des multi-ensembles à l’aide de listes d’association\\}

\paragraph{Question 1}

On définit ici un multi-ensemble comme une liste de couples (élément, cardinalité), de la manière suivante :

\begin{coqdoccode}
\coqdocemptyline
\coqdocnoindent
\coqdockw{Definition} \coqdocvar{multiset} := \coqdocvar{list} (\coqdocvar{T}\ensuremath{\times}\coqdocvar{nat}).\coqdoceol
\coqdocemptyline
\coqdocemptyline
\end{coqdoccode}

\noindent Voici les signatures des différentes fonctions demandées dans l'énoncé :

\noindent\begin{coqdoccode}
\coqdocemptyline
\coqdocnoindent
\coqdocvar{empty} :
\coqdocvar{multiset}
\end{coqdoccode}

\noindent\begin{coqdoccode}
\coqdocemptyline
\coqdocnoindent
\coqdocvar{singleton} :
\coqdocvar{T} \ensuremath{\rightarrow}
\coqdocvar{multiset}
\end{coqdoccode}

\noindent\begin{coqdoccode}
\coqdocemptyline
\coqdocnoindent
\coqdocvar{add} :
\coqdocvar{T} \ensuremath{\rightarrow}
\coqdocvar{nat} \ensuremath{\rightarrow}
\coqdocvar{multiset} \ensuremath{\rightarrow}
\coqdocvar{multiset}
\end{coqdoccode}

\noindent\begin{coqdoccode}
\coqdocemptyline
\coqdocnoindent
\coqdocvar{member} :
\coqdocvar{T} \ensuremath{\rightarrow}
\coqdocvar{multiset} \ensuremath{\rightarrow}
\coqdocvar{bool}
\end{coqdoccode}

\noindent\begin{coqdoccode}
\coqdocemptyline
\coqdocnoindent
\coqdocvar{union} :
\coqdocvar{multiset} \ensuremath{\rightarrow}
\coqdocvar{multiset} \ensuremath{\rightarrow}
\coqdocvar{multiset}
\end{coqdoccode}

\noindent\begin{coqdoccode}
\coqdocemptyline
\coqdocnoindent
\coqdocvar{multiplicity} :
\coqdocvar{T} \ensuremath{\rightarrow}
\coqdocvar{multiset} \ensuremath{\rightarrow}
\coqdocvar{nat}
\end{coqdoccode}

\noindent\begin{coqdoccode}
\coqdocemptyline
\coqdocnoindent
\coqdocvar{removeOne} :
\coqdocvar{T} \ensuremath{\rightarrow}
\coqdocvar{multiset} \ensuremath{\rightarrow}
\coqdocvar{multiset}
\end{coqdoccode}

\noindent\begin{coqdoccode}
\coqdocemptyline
\coqdocnoindent
\coqdocvar{removeAll} :
\coqdocvar{T} \ensuremath{\rightarrow}
\coqdocvar{multiset} \ensuremath{\rightarrow}
\coqdocvar{multiset}
\end{coqdoccode}

\paragraph{Question 2}

\paragraph{Question 2. a)}

On définit le prédicat \texttt{InMultiset} de la manière suivante :

\noindent\begin{coqdoccode}
\coqdocemptyline
\coqdocnoindent
\coqdockw{Inductive} \coqdocvar{InMultiset} (\coqdocvar{x}:\coqdocvar{T}) (\coqdocvar{l}:\coqdocvar{multiset}) : \coqdockw{Prop} := \coqdoceol
\coqdocindent{1.00em}
\ensuremath{|} \coqdocvar{inMultiset\_intro} : \coqdocvar{member} \coqdocvar{x} \coqdocvar{l} = \coqdocvar{true} \ensuremath{\rightarrow} \coqdocvar{InMultiset} \coqdocvar{x} \coqdocvar{l}.\coqdoceol
\coqdocemptyline
\coqdocemptyline
\end{coqdoccode}

\paragraph{Question 2. b)}

On définit le prédicat \texttt{wf} de la manière suivante :

\noindent\begin{coqdoccode}
\coqdocemptyline
\coqdocnoindent
\coqdockw{Inductive} \coqdockw{wf} (\coqdocvar{l}: \coqdocvar{multiset}) : \coqdockw{Prop} :=\coqdoceol
\coqdocindent{1.00em}
\ensuremath{|} \coqdocvar{wf\_intro} : (\coqdockw{\ensuremath{\forall}} \coqdocvar{x}, (\coqdocvar{InMultiset} \coqdocvar{x} (\coqdocvar{removeAll} \coqdocvar{x} \coqdocvar{l}) \ensuremath{\rightarrow} \coqdocvar{False}) \ensuremath{\land} (\coqdocvar{member} \coqdocvar{x} \coqdocvar{l} = \coqdocvar{true} \ensuremath{\rightarrow} (\coqdocvar{multiplicity} \coqdocvar{x} \coqdocvar{l}) > 0)) \ensuremath{\rightarrow} \coqdockw{wf} \coqdocvar{l}.\coqdoceol
\coqdocemptyline
\end{coqdoccode}

\paragraph{Question 2. c)}

On démontre maintenant la bonne formation de nos différentes fonctions. 2 lemmes intermédiaires ont été créés.\\
Nous ne sommes pas parvenus à démontrer cette propriété pour 4 fonctions :

\noindent\begin{coqdoccode}
\coqdocemptyline
\coqdocnoindent
\coqdockw{Theorem} \coqdocvar{empty\_wf} : \coqdockw{wf} \coqdocvar{empty}.\coqdoceol
\coqdocemptyline
\end{coqdoccode}

\begin{coqdoccode}
\coqdocnoindent
\coqdockw{Theorem} \coqdocvar{singleton\_wf} : \coqdockw{\ensuremath{\forall}} (\coqdocvar{x}:\coqdocvar{T}), \coqdockw{wf} (\coqdocvar{singleton} \coqdocvar{x}).\coqdoceol
\coqdocemptyline
\end{coqdoccode}

\begin{coqdoccode}
\coqdocnoindent
\coqdockw{Lemma} \coqdocvar{plus\_n\_0} : \coqdockw{\ensuremath{\forall}} \coqdocvar{n}, \coqdocvar{n} + \coqdocvar{O} = \coqdocvar{n}.\coqdoceol
\coqdocemptyline
\end{coqdoccode}

\begin{coqdoccode}
\coqdocnoindent
\coqdockw{Lemma} \coqdocvar{wf\_plus\_n} : \coqdockw{\ensuremath{\forall}} \coqdocvar{t} \coqdocvar{n0} \coqdocvar{n} \coqdocvar{l}, \coqdockw{wf} ((\coqdocvar{t}, \coqdocvar{n0}) :: \coqdocvar{l}) \ensuremath{\rightarrow} \coqdockw{wf} ((\coqdocvar{t}, \coqdocvar{n0}+\coqdocvar{n}) :: \coqdocvar{l}).\coqdoceol
\coqdocemptyline
\end{coqdoccode}

\begin{coqdoccode}
\coqdocnoindent
\textcolor{red}{
\coqdockw{Theorem} \coqdocvar{add\_wf} : \coqdockw{\ensuremath{\forall}} (\coqdocvar{x}:\coqdocvar{T}) (\coqdocvar{n}:\coqdocvar{nat}) (\coqdocvar{l}: \coqdocvar{multiset}), \coqdockw{wf} \coqdocvar{l} \ensuremath{\rightarrow} \coqdockw{wf} (\coqdocvar{add} \coqdocvar{x} \coqdocvar{n} \coqdocvar{l}).\coqdoceol
\coqdocemptyline
}
\end{coqdoccode}

\begin{coqdoccode}
\coqdocnoindent
\textcolor{red}{
\coqdockw{Theorem} \coqdocvar{union\_wf} : \coqdockw{\ensuremath{\forall}} (\coqdocvar{l}: \coqdocvar{multiset}) (\coqdocvar{l'}: \coqdocvar{multiset}), \coqdockw{wf} \coqdocvar{l} \ensuremath{\rightarrow} \coqdockw{wf} \coqdocvar{l'} \ensuremath{\rightarrow} \coqdockw{wf} (\coqdocvar{union} \coqdocvar{l} \coqdocvar{l'}).\coqdoceol
\coqdocemptyline
}
\end{coqdoccode}

\begin{coqdoccode}
\coqdocnoindent
\textcolor{red}{
\coqdockw{Theorem} \coqdocvar{removeOne\_wf}: \coqdockw{\ensuremath{\forall}} (\coqdocvar{x}: \coqdocvar{T}) (\coqdocvar{l}: \coqdocvar{multiset}), \coqdockw{wf} \coqdocvar{l} \ensuremath{\rightarrow} \coqdockw{wf} (\coqdocvar{removeOne} \coqdocvar{x} \coqdocvar{l}).\coqdoceol
\coqdocemptyline
}
\end{coqdoccode}

\begin{coqdoccode}
\coqdocnoindent
\textcolor{red}{
\coqdockw{Theorem} \coqdocvar{removeAll\_wf}: \coqdockw{\ensuremath{\forall}} (\coqdocvar{x}: \coqdocvar{T}) (\coqdocvar{l}: \coqdocvar{multiset}), \coqdockw{wf} \coqdocvar{l} \ensuremath{\rightarrow} \coqdockw{wf} (\coqdocvar{removeAll} \coqdocvar{x} \coqdocvar{l}).\coqdoceol
\coqdocemptyline
}
\end{coqdoccode}

\paragraph{Question 3}

On démontre maintenant les différents théorèmes indiqués dans l'énoncé. 1 lemme intermédiaire a été créé.\\
Nous ne sommes pas parvenus à démontrer les deux derniers théorèmes :

\noindent\begin{coqdoccode}
\coqdocemptyline
\coqdocnoindent
\coqdockw{Theorem} \coqdocvar{proof\_1\_1} : \coqdockw{\ensuremath{\forall}} (\coqdocvar{x} : \coqdocvar{T}), \ensuremath{\lnot}\coqdocvar{InMultiset} \coqdocvar{x} \coqdocvar{empty}.\coqdoceol
\coqdocemptyline
\coqdocnoindent
\coqdockw{Theorem} \coqdocvar{proof\_2\_1} : \coqdockw{\ensuremath{\forall}} \coqdocvar{x} \coqdocvar{y} , \coqdocvar{InMultiset} \coqdocvar{y} (\coqdocvar{singleton} \coqdocvar{x}) \ensuremath{\leftrightarrow} \coqdocvar{x} = \coqdocvar{y}.\coqdoceol
\coqdocemptyline
\coqdocnoindent
\coqdockw{Theorem} \coqdocvar{proof\_3\_1} :\coqdockw{\ensuremath{\forall}} \coqdocvar{x}, \coqdocvar{multiplicity} \coqdocvar{x} (\coqdocvar{singleton} \coqdocvar{x}) = 1.\coqdoceol
\coqdocemptyline
\coqdocnoindent
\coqdockw{Theorem} \coqdocvar{proof\_4\_1} : \coqdockw{\ensuremath{\forall}} \coqdocvar{x} \coqdocvar{s}, \coqdockw{wf} \coqdocvar{s} \ensuremath{\rightarrow} (\coqdocvar{member} \coqdocvar{x} \coqdocvar{s} = \coqdocvar{true} \ensuremath{\leftrightarrow} \coqdocvar{InMultiset} \coqdocvar{x} \coqdocvar{s}).\coqdoceol
\coqdocemptyline
\coqdocnoindent
\coqdockw{Theorem} \coqdocvar{proof\_5\_1} : \coqdockw{\ensuremath{\forall}} \coqdocvar{x} \coqdocvar{n} \coqdocvar{s}, \coqdocvar{n} > 0 \ensuremath{\rightarrow} \coqdocvar{InMultiset} \coqdocvar{x} (\coqdocvar{add} \coqdocvar{x} \coqdocvar{n} \coqdocvar{s}).\coqdoceol
\coqdocemptyline
\coqdocnoindent
\coqdockw{Theorem} \coqdocvar{proof\_6\_1} : \coqdockw{\ensuremath{\forall}} \coqdocvar{x} \coqdocvar{n} \coqdocvar{y} \coqdocvar{s}, \coqdocvar{x} \ensuremath{\not=} \coqdocvar{y} \ensuremath{\rightarrow} (\coqdocvar{InMultiset} \coqdocvar{y} (\coqdocvar{add} \coqdocvar{x} \coqdocvar{n} \coqdocvar{s}) \ensuremath{\leftrightarrow} \coqdocvar{InMultiset} \coqdocvar{y} \coqdocvar{s}).\coqdoceol
\coqdocemptyline
\end{coqdoccode}

\begin{coqdoccode}
\coqdocnoindent
\coqdockw{Lemma} \coqdocvar{proof\_7\_1\_1}: \coqdockw{\ensuremath{\forall}} \coqdocvar{x} \coqdocvar{s}, \coqdocvar{multiplicity} \coqdocvar{x} \coqdocvar{s} \ensuremath{\not=} 0 \ensuremath{\rightarrow} \coqdocvar{InMultiset} \coqdocvar{x} \coqdocvar{s}.\coqdoceol
\coqdocemptyline
\coqdocnoindent
\coqdockw{Theorem} \coqdocvar{proof\_7\_1\_2} : \coqdockw{\ensuremath{\forall}} \coqdocvar{x} \coqdocvar{s}, \coqdockw{wf} \coqdocvar{s} \ensuremath{\rightarrow} (\coqdocvar{multiplicity} \coqdocvar{x} \coqdocvar{s} = 0 \ensuremath{\leftrightarrow} \ensuremath{\lnot}\coqdocvar{InMultiset} \coqdocvar{x} \coqdocvar{s}).\coqdoceol
\coqdocemptyline
\coqdocnoindent
\coqdockw{Theorem} \coqdocvar{proof\_8\_1} : \coqdockw{\ensuremath{\forall}} \coqdocvar{x} \coqdocvar{n} \coqdocvar{s}, \coqdocvar{multiplicity} \coqdocvar{x} (\coqdocvar{add} \coqdocvar{x} \coqdocvar{n} \coqdocvar{s}) = \coqdocvar{n} + (\coqdocvar{multiplicity} \coqdocvar{x} \coqdocvar{s}).\coqdoceol
\coqdocemptyline
\end{coqdoccode}

\begin{coqdoccode}
\coqdocnoindent
\textcolor{red}{
\coqdockw{Theorem} \coqdocvar{proof\_9\_1} : \coqdockw{\ensuremath{\forall}} \coqdocvar{x} \coqdocvar{n} \coqdocvar{y} \coqdocvar{s}, \coqdocvar{x} \ensuremath{\not=} \coqdocvar{y} \ensuremath{\rightarrow} \coqdockw{wf} \coqdocvar{s} \ensuremath{\rightarrow}\coqdocvar{multiplicity} \coqdocvar{y} (\coqdocvar{add} \coqdocvar{x} \coqdocvar{n} \coqdocvar{s}) = \coqdocvar{multiplicity} \coqdocvar{y} \coqdocvar{s}.\coqdoceol
\coqdocemptyline
}
\end{coqdoccode}

\begin{coqdoccode}
\coqdocnoindent
\textcolor{red}{
\coqdockw{Theorem} \coqdocvar{proof\_10\_1} : \coqdockw{\ensuremath{\forall}} \coqdocvar{s} \coqdocvar{t} \coqdocvar{x}, \coqdockw{wf} \coqdocvar{s} \ensuremath{\rightarrow} \coqdockw{wf} \coqdocvar{t} ->(\coqdocvar{InMultiset} \coqdocvar{x} (\coqdocvar{union} \coqdocvar{s} \coqdocvar{t}) \ensuremath{\leftrightarrow} \coqdocvar{InMultiset} \coqdocvar{x} \coqdocvar{s} \ensuremath{\lor} \coqdocvar{InMultiset} \coqdocvar{x} \coqdocvar{t}).\coqdoceol
\coqdocemptyline
}
\end{coqdoccode}

\paragraph{Question 4}

On démontre maintenant quelques théorèmes supplémentaires sur \texttt{removeOne} et \texttt{removeAll}.

\noindent\begin{coqdoccode}
\coqdocemptyline
\coqdocnoindent
\coqdockw{Theorem} \coqdocvar{proof\_11\_1} : \coqdockw{\ensuremath{\forall}} \coqdocvar{x}, \coqdocvar{multiplicity} \coqdocvar{x} (\coqdocvar{removeOne} \coqdocvar{x} (\coqdocvar{singleton} \coqdocvar{x})) = 0.\coqdoceol
\coqdocemptyline
\coqdocnoindent
\coqdockw{Theorem} \coqdocvar{proof\_12\_1} : \coqdockw{\ensuremath{\forall}} \coqdocvar{x}, \coqdocvar{multiplicity} \coqdocvar{x} (\coqdocvar{removeAll} \coqdocvar{x} (\coqdocvar{singleton} \coqdocvar{x})) = 0.\coqdoceol
\coqdocemptyline
\coqdocnoindent
\coqdockw{Theorem} \coqdocvar{proof\_13\_1} : \coqdockw{\ensuremath{\forall}} \coqdocvar{x} \coqdocvar{l} \coqdocvar{n}, \coqdocvar{multiplicity} \coqdocvar{x} \coqdocvar{l} = \coqdocvar{n} \ensuremath{\rightarrow} \coqdocvar{n} > 1 \ensuremath{\rightarrow} \coqdocvar{multiplicity} \coqdocvar{x} (\coqdocvar{removeOne} \coqdocvar{x} \coqdocvar{l}) = \coqdocvar{n}-1.\coqdoceol
\coqdocemptyline
\coqdocnoindent
\coqdockw{Theorem} \coqdocvar{proof\_14\_1} : \coqdockw{\ensuremath{\forall}} \coqdocvar{x} \coqdocvar{l}, \coqdockw{wf} \coqdocvar{l} \ensuremath{\rightarrow} \~{}(\coqdocvar{InMultiset} \coqdocvar{x} (\coqdocvar{removeAll} \coqdocvar{x} \coqdocvar{l})).\coqdoceol
\coqdocemptyline
\coqdocnoindent
\coqdockw{Theorem} \coqdocvar{proof\_15\_1} : \coqdockw{\ensuremath{\forall}} \coqdocvar{x} \coqdocvar{l}, \coqdocvar{multiplicity} \coqdocvar{x} \coqdocvar{l} > 1 \ensuremath{\rightarrow} \coqdocvar{InMultiset} \coqdocvar{x} (\coqdocvar{removeOne} \coqdocvar{x} \coqdocvar{l}).\coqdoceol
\coqdocemptyline
\coqdocemptyline
\end{coqdoccode}

\newpage

\subsection{Implantation Fonctionnelle des multi-ensembles\\}

\paragraph{Question 1}

On définit ici un multi-ensemble comme une fonction qui encode les multiplicités, de la manière suivante :

\noindent\begin{coqdoccode}
\coqdocemptyline
\coqdocnoindent
\coqdockw{Definition} \coqdocvar{multiset\_2} := \coqdocvar{T} \ensuremath{\rightarrow} \coqdocvar{nat}.\coqdoceol
\coqdocnoindent
\coqdocemptyline
\end{coqdoccode}

\noindent Voici les signatures des différentes fonctions demandées dans l'énoncé :

\noindent\begin{coqdoccode}
\coqdocemptyline
\coqdocnoindent
\coqdocvar{empty\_2} :
\coqdocvar{multiset\_2}
\end{coqdoccode}

\noindent\begin{coqdoccode}
\coqdocemptyline
\coqdocnoindent
\coqdocvar{singleton\_2} :
\coqdocvar{T} \ensuremath{\rightarrow}
\coqdocvar{multiset\_2}
\end{coqdoccode}

\noindent\begin{coqdoccode}
\coqdocemptyline
\coqdocnoindent
\coqdocvar{add\_2} :
\coqdocvar{T} \ensuremath{\rightarrow}
\coqdocvar{nat} \ensuremath{\rightarrow}
\coqdocvar{multiset\_2} \ensuremath{\rightarrow}
\coqdocvar{multiset\_2}
\end{coqdoccode}

\noindent\begin{coqdoccode}
\coqdocemptyline
\coqdocnoindent
\coqdocvar{member\_2} :
\coqdocvar{T} \ensuremath{\rightarrow}
\coqdocvar{multiset\_2} \ensuremath{\rightarrow}
\coqdocvar{bool}
\end{coqdoccode}

\noindent\begin{coqdoccode}
\coqdocemptyline
\coqdocnoindent
\coqdocvar{union\_2} :
\coqdocvar{multiset\_2} \ensuremath{\rightarrow}
\coqdocvar{multiset\_2} \ensuremath{\rightarrow}
\coqdocvar{multiset\_2}
\end{coqdoccode}

\noindent\begin{coqdoccode}
\coqdocemptyline
\coqdocnoindent
\coqdocvar{multiplicity\_2} :
\coqdocvar{T} \ensuremath{\rightarrow}
\coqdocvar{multiset\_2} \ensuremath{\rightarrow}
\coqdocvar{nat}
\end{coqdoccode}

\noindent\begin{coqdoccode}
\coqdocemptyline
\coqdocnoindent
\coqdocvar{removeOne\_2} :
\coqdocvar{T} \ensuremath{\rightarrow}
\coqdocvar{multiset\_2} \ensuremath{\rightarrow}
\coqdocvar{multiset\_2}
\end{coqdoccode}

\noindent\begin{coqdoccode}
\coqdocemptyline
\coqdocnoindent
\coqdocvar{removeAll\_2} :
\coqdocvar{T} \ensuremath{\rightarrow}
\coqdocvar{multiset\_2} \ensuremath{\rightarrow}
\coqdocvar{multiset\_2}
\end{coqdoccode}

\paragraph{Question 2}

On définit le prédicat \texttt{InMultiset\_2} de la manière suivante :\\

\begin{coqdoccode}
\coqdocnoindent
\coqdockw{Inductive} \coqdocvar{InMultiset\_2} (\coqdocvar{x}:\coqdocvar{T}) (\coqdocvar{l}:\coqdocvar{multiset\_2}) : \coqdockw{Prop} := \coqdoceol
\coqdocindent{1.00em}
\ensuremath{|} \coqdocvar{inMultiset\_2\_intro} : \coqdocvar{member\_2} \coqdocvar{x} \coqdocvar{l} = \coqdocvar{true} \ensuremath{\rightarrow} \coqdocvar{InMultiset\_2} \coqdocvar{x} \coqdocvar{l}.\coqdoceol
\coqdocemptyline
\end{coqdoccode}

\paragraph{Question 3}

On démontre maintenant les différents théorèmes indiqués dans l’énoncé ainsi que ceux sur les fonctions \texttt{removeOne} et \texttt{removeAll}. 1 lemme intermédiaire a été créé.

\noindent\begin{coqdoccode}
\coqdocemptyline
\coqdocnoindent
\coqdockw{Theorem} \coqdocvar{proof\_1\_2} : \coqdockw{\ensuremath{\forall}} (\coqdocvar{x} : \coqdocvar{T}), \ensuremath{\lnot}\coqdocvar{InMultiset\_2} \coqdocvar{x} \coqdocvar{empty\_2}.\coqdoceol
\coqdocemptyline
\coqdocnoindent
\coqdockw{Theorem} \coqdocvar{proof\_2\_2} : \coqdockw{\ensuremath{\forall}} \coqdocvar{x} \coqdocvar{y} , \coqdocvar{InMultiset\_2} \coqdocvar{y} (\coqdocvar{singleton\_2} \coqdocvar{x}) \ensuremath{\leftrightarrow} \coqdocvar{x} = \coqdocvar{y}.\coqdoceol
\coqdocemptyline
\coqdocnoindent
\coqdockw{Theorem} \coqdocvar{proof\_3\_2} :\coqdockw{\ensuremath{\forall}} \coqdocvar{x}, \coqdocvar{multiplicity\_2} \coqdocvar{x} (\coqdocvar{singleton\_2} \coqdocvar{x}) = 1.\coqdoceol
\coqdocemptyline
\coqdocnoindent
\coqdockw{Theorem} \coqdocvar{proof\_4\_2} : \coqdockw{\ensuremath{\forall}} \coqdocvar{x} \coqdocvar{s}, \coqdocvar{member\_2} \coqdocvar{x} \coqdocvar{s} = \coqdocvar{true} \ensuremath{\leftrightarrow} \coqdocvar{InMultiset\_2} \coqdocvar{x} \coqdocvar{s}.\coqdoceol
\coqdocemptyline
\coqdocnoindent
\coqdockw{Theorem} \coqdocvar{proof\_5\_2} : \coqdockw{\ensuremath{\forall}} \coqdocvar{x} \coqdocvar{n} \coqdocvar{s}, \coqdocvar{n} > 0 \ensuremath{\rightarrow} \coqdocvar{InMultiset\_2} \coqdocvar{x} (\coqdocvar{add\_2} \coqdocvar{x} \coqdocvar{n} \coqdocvar{s}).\coqdoceol
\coqdocemptyline
\coqdocnoindent
\coqdockw{Theorem} \coqdocvar{proof\_6\_2} : \coqdockw{\ensuremath{\forall}} \coqdocvar{x} \coqdocvar{n} \coqdocvar{y} \coqdocvar{s}, \coqdocvar{x} \ensuremath{\not=} \coqdocvar{y} \ensuremath{\rightarrow} (\coqdocvar{InMultiset\_2} \coqdocvar{y} (\coqdocvar{add\_2} \coqdocvar{x} \coqdocvar{n} \coqdocvar{s}) \ensuremath{\leftrightarrow} \coqdocvar{InMultiset\_2} \coqdocvar{y} \coqdocvar{s}).\coqdoceol
\coqdocemptyline
\coqdocnoindent
\coqdockw{Lemma} \coqdocvar{proof\_7\_2\_1}: \coqdockw{\ensuremath{\forall}} \coqdocvar{x} \coqdocvar{s}, \coqdocvar{s} \coqdocvar{x} \ensuremath{\not=} 0 \ensuremath{\rightarrow} \coqdocvar{InMultiset\_2} \coqdocvar{x} \coqdocvar{s}.\coqdoceol
\coqdocemptyline
\coqdocnoindent
\coqdockw{Theorem} \coqdocvar{proof\_7\_2\_2} : \coqdockw{\ensuremath{\forall}} \coqdocvar{x} \coqdocvar{s}, \coqdocvar{multiplicity\_2} \coqdocvar{x} \coqdocvar{s} = 0 \ensuremath{\leftrightarrow} \ensuremath{\lnot}\coqdocvar{InMultiset\_2} \coqdocvar{x} \coqdocvar{s}.\coqdoceol
\coqdocemptyline
\coqdocnoindent
\coqdockw{Theorem} \coqdocvar{proof\_8\_2} : \coqdockw{\ensuremath{\forall}} \coqdocvar{x} \coqdocvar{n} \coqdocvar{s}, \coqdocvar{multiplicity\_2} \coqdocvar{x} (\coqdocvar{add\_2} \coqdocvar{x} \coqdocvar{n} \coqdocvar{s}) = \coqdocvar{n} + (\coqdocvar{multiplicity\_2} \coqdocvar{x} \coqdocvar{s}).\coqdoceol
\coqdocemptyline
\coqdocnoindent
\coqdockw{Theorem} \coqdocvar{proof\_9\_2} : \coqdockw{\ensuremath{\forall}} \coqdocvar{x} \coqdocvar{n} \coqdocvar{y} \coqdocvar{s}, \coqdocvar{x} \ensuremath{\not=} \coqdocvar{y} \ensuremath{\rightarrow} \coqdocvar{multiplicity\_2} \coqdocvar{y} (\coqdocvar{add\_2} \coqdocvar{x} \coqdocvar{n} \coqdocvar{s}) = \coqdocvar{multiplicity\_2} \coqdocvar{y} \coqdocvar{s}.\coqdoceol
\coqdocemptyline
\coqdocnoindent
\coqdockw{Theorem} \coqdocvar{proof\_10\_2} : \coqdockw{\ensuremath{\forall}} \coqdocvar{s} \coqdocvar{t} \coqdocvar{x}, (\coqdocvar{InMultiset\_2} \coqdocvar{x} (\coqdocvar{union\_2} \coqdocvar{s} \coqdocvar{t}) \ensuremath{\leftrightarrow} \coqdocvar{InMultiset\_2} \coqdocvar{x} \coqdocvar{s} \ensuremath{\lor} \coqdocvar{InMultiset\_2} \coqdocvar{x} \coqdocvar{t}).\coqdoceol
\coqdocemptyline
\coqdocnoindent
\coqdockw{Theorem} \coqdocvar{proof\_11\_2} : \coqdockw{\ensuremath{\forall}} \coqdocvar{x}, \coqdocvar{multiplicity\_2} \coqdocvar{x} (\coqdocvar{removeOne\_2} \coqdocvar{x} (\coqdocvar{singleton\_2} \coqdocvar{x})) = 0.\coqdoceol
\coqdocemptyline
\coqdocnoindent
\coqdockw{Theorem} \coqdocvar{proof\_12\_2} : \coqdockw{\ensuremath{\forall}} \coqdocvar{x}, \coqdocvar{multiplicity\_2} \coqdocvar{x} (\coqdocvar{removeAll\_2} \coqdocvar{x} (\coqdocvar{singleton\_2} \coqdocvar{x})) = 0.\coqdoceol
\coqdocemptyline
\coqdocnoindent
\coqdockw{Theorem} \coqdocvar{proof\_13\_2} : \coqdockw{\ensuremath{\forall}} \coqdocvar{x} \coqdocvar{l} \coqdocvar{n}, \coqdocvar{multiplicity\_2} \coqdocvar{x} \coqdocvar{l} = \coqdocvar{n} \ensuremath{\rightarrow} \coqdocvar{n} > 1 \ensuremath{\rightarrow} \coqdocvar{multiplicity\_2} \coqdocvar{x} (\coqdocvar{removeOne\_2} \coqdocvar{x} \coqdocvar{l}) = \coqdocvar{n}-1.\coqdoceol
\coqdocemptyline
\coqdocnoindent
\coqdockw{Theorem} \coqdocvar{proof\_14\_2} : \coqdockw{\ensuremath{\forall}} \coqdocvar{x} \coqdocvar{l}, \~{}(\coqdocvar{InMultiset\_2} \coqdocvar{x} (\coqdocvar{removeAll\_2} \coqdocvar{x} \coqdocvar{l})).\coqdoceol
\coqdocemptyline
\coqdocnoindent
\coqdockw{Theorem} \coqdocvar{proof\_15\_2} : \coqdockw{\ensuremath{\forall}} \coqdocvar{x} \coqdocvar{l}, \coqdocvar{multiplicity\_2} \coqdocvar{x} \coqdocvar{l} > 1 \ensuremath{\rightarrow} \coqdocvar{InMultiset\_2} \coqdocvar{x} (\coqdocvar{removeOne\_2} \coqdocvar{x} \coqdocvar{l}).\coqdoceol
\end{coqdoccode}
\end{document}
